\section{Presentation of the method}


The proposed algorithm \cite{claici_stochastic_2018} boils down to two key steps :
\begin{enumerate}
    \item \label{step1} \textbf{Stochastic Gradient Ascent for Potentials:} With $\lbrace x_i \rbrace$ fixed, a stochastic gradient ascent is employed to optimize the potentials $\lbrace \phi_j^i \rbrace$, leveraging the concavity of F \eqref{eq:obj} in $\phi_j$.
    \item \textbf{Fixed-Point Iteration for Positions:} With $\lbrace \phi_j^i \rbrace$ fixed, a single fixed point iteration is performed to update the barycenter grid $\lbrace x_i \rbrace$. This update is facilitated by a closed-form expression for the zeroed values of the gradient, i.e., $$\frac{\partial F}{\partial x_i} = 0$$
\end{enumerate}

\subsection{Stochastic Gradient Ascent for Potentials}

For step \ref{step1}, we note that $F$ is concave in the potentials $\phi_j$. Indeed, we have shown in the course \cite{peyre_computational_2020} that the $c$-transform is concave for the euclidean cost $d(x, y)^2 = \norm{x - y}_2^2$. 
To get the gradient, we need to compute the quantities 


\begin{align*}
    a^i_j &= \int_{V_{\phi_j}^i} d\mu_j(y) & b_j^i &= \int_{V_{\phi_j}^i} y\, d\mu_j(y) \\
     &= \E_{y \sim \mu_j}\left[ \mathbf{1}_{y\in V_{\phi_j}^i}\right] & & =\E_{y \sim \mu_j}\left[ y . \ones_{y\in V_{\phi_j^i}}\right] \\
\end{align*}

where $\ones$ indicates the indicator function of a set. In the derivative of \eqref{eq:obj}, the \textit{power cell} $V_{\phi_j}^i$ of point $x_i$ comes into play and is defined as follows \blue{link with voronoi cell !!!}
$$
V_{\phi}^i = \ens{x\in\mathcal{X}}{d(x, x_i)^2 - \phi_i \leq d(x, x_{i'})^2 - \phi_{i'}, \forall i' }
$$

The computation of $b_j^i$ is given in the algorithm \ref{montecarlo}, the computation of $a_j^i$ is similar. 

\begin{algorithm}[H]
    \caption{Estimate $a_j^i$ and $b_j^i$ by Monte Carlo approximation}\label{montecarlo}
    \begin{algorithmic}[1] % enable line numbers
        \REQUIRE iid samples $\lbrace Y\rbrace_{k=1}^K \sim \mu_j$ 
        \FOR {$y$ in $Y$} 
            \IF {$y\in V_{\phi_j}^i$ }
                \STATE $b_j^i \gets b_j^i + y$
            \ENDIF 
        \ENDFOR
        \STATE $b_j \gets \frac{1}{K} b_j$
        \RETURN $b_j^i$
    \end{algorithmic}
\end{algorithm}

The well known gradient ascent step is given by $$ \phi_j^{(l+1)} = \phi_j^{(l)} + \alpha \frac{\partial F}{\partial \phi_j}(\phi_j^{(l)}) $$
at iteration $l$, where $\lambda$ is the step size. To improve performance, the authors apply the Nesterov acceleration. The \textit{ascent step} is given in algorithm \ref{ascent}.

\begin{algorithm}[H]
    \caption{Ascent step}\label{ascent}
    \begin{algorithmic}[1]
        \FOR {$j = 1, \dots, J$}
        \STATE $z^{(0)} \gets 0$
        \WHILE {$\norm{\frac{\widehat{\partial F}}{\partial \phi_j}} > \epsilon $}
        \STATE Compute $\hat{a}_j^i$ thanks to algorithm \ref{montecarlo}
        \STATE $z^{l+1} \gets \beta z^{(l)} + \widehat{\partial_{\phi_j}F}(\phi_j)$ \COMMENT{Nesterov acceleration}
        \STATE $\phi_j^{(l+1)} \gets \phi_j^{(l)} + \alpha z^{(l+1)}$
        \ENDWHILE
        \ENDFOR
    \end{algorithmic}
\end{algorithm}
In algorithm \ref{ascent}, $\widehat{\partial_{\phi_j}F}$ denote the stochastic approximation of the gradient of $F$ with respect to $\phi_j$. 

\subsection{Fixed Point Iteration for Positions}

The update of the point $\lbrace x_i \rbrace$ is given by 
$$
\frac{\partial F}{\partial x_i} = 0 \implies x_i = \frac{\sum_{j=1}^J b_j^i }{\sum_{j=1}^{J}a_j^i}
$$
This step boils down to a simple fixed point iteration and is denoted as the \textit{snap step}.
The algorithm ref{?} summarize the whole procedure. 

\begin{algorithm}
    \caption{Ascent and Snap algorithm for commputing Stochastic Wasserstein Barycenters}\label{ascent_snap}
    \begin{algorithmic}[1]
        \FOR {$t = 1, \dots, T$}
            \STATE Update $\lbrace \phi_j \rbrace$ with algorithm \ref{ascent} \COMMENT{Ascent step}
            \STATE Compute $\hat{b}_j^i$ with algorithm \ref{montecarlo} \COMMENT{Snap step}
            \FOR {$x_i \in \Sigma$}  
                \STATE $x_i \gets \frac{\sum_{j=1}^J \hat{b}_j^i }{\sum_{j=1}^{J}\hat{a}_j^i}$
            \ENDFOR
        \ENDFOR
        \RETURN Opitimized barycenter support $\Sigma^* = \lbrace x_i \rbrace_{i=1}^M$
    \end{algorithmic}
\end{algorithm}