\section{Connection with the course}

This study directly aligns with the course material. The article \cite{claici_stochastic_2018} is based on the notion of Wasserstein distance introduced during the course. 

Additionally, we have conducted a comparative analysis of the proposed method with two approaches presented in the course.

We employed the Sinkhorn algorithm to calculate the barycenter of two measures in both 1-D and 2-D. In the case of 1-D, we have also computed the McCann's interpolation between two continuous measures, as depicted in Figure \ref{fig:sinkhorn_1D_2skew}.

Regarding the results, we observed the same influence of the regularization parameter $\epsilon$ on the Sinkhorn algorithm, as illustrated in Figure \ref{fig:sinkhorn_1D_2skew}. 